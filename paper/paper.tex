\documentclass[11pt,a4paper]{article}

\usepackage[margin=1in]{geometry}

\title{Cellular Automata and Computational Universality}

\begin{document}

\maketitle

\begin{center}
    Thomas Archbold \\
    University of Warwick \\
    \texttt{T.Archbold@warwick.ac.uk}
\end{center}

\begin{abstract}
    Cellular automata are discrete models with the ability to not only give rise
    to beautiful, intricate patterns, but also to be used as powerful tools of
    computation, with applications in cryptography, error-correction coding, and
    simulation of computer processors, to name a few. They also raise profound
    questions about the nature of our reality, asking whether our universe could
    be one such automaton. This paper provides a discussion into these automata,
    exploring the various power and limitations of a number of specific rule
    sets, covering John Conway's well-known ``Game of Life'' to the more obscure
    ``Langton's ant'' and ``Wireworld''. In particular, it explores the notion
    of computational universality, or Turing completeness, an automaton's
    ability to simulate any conceivable computation, and considers their
    potential in the context of solving two specific problems, the Firing Squad
    Synchronization Problem and the Majority Problem. Existing solutions to
    these problems are explored, and their existing avenues for optimization are
    discussed. In order to fully appreciate the complex structures that can
    arise from such simple beginnings, this project also presents software to
    visualize and probe further into the nature of the automata highlighted.
\end{abstract}

\end{document}
